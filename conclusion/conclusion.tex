\chapter{Conclusion and outlook}
\label{chap:conclusion}

The work in this dissertation has approached the topic of true polar wander
from several directions, including scaling, numerics, and data analysis.

In Chapter~\ref{chap:tpw_rate} we analyzed TPW from the perspective of fluid dynamics.
The rotating planet is most stable when the spin axis coincides with the axis of the largest moment of inertia.
TPW is primarily a balance bewteen the generation of anomalies in the moment of inertia tensor
via convection and their decay via movement of the spin axis.
Our scaling showed that the primary nondimensional numbers controling TPW are the Rayleigh number $\mathrm{Ra}$,
and the Froude number $m$, defined by the ratio of centrifugal to gravitational forces.

The Froude number sets the size of the rotational bulge, which acts
as the brakes on TPW. The dependence on the Rayleigh number is more complicated:
at high $\mathrm{Ra}$ the flow becomes more chaotic, and the total power in the degree-two
part of the density field (which is the part that controls TPW) goes down.
However, at high $\mathrm{Ra}$ the characteristic response time for TPW also goes down,
and these two effects largely cancel. Additionally, at higher $\mathrm{Ra}$,
the perturbations to the moment of inertia can produce larger angular differences
between the axis of the maximum moment of inertia and the spin axis,
which results in faster rates of TPW. The net effect of higher $\mathrm{Ra}$,
and the correspondingly more vigorous convection, is that TPW events become more likely.

\citet{gold1955instability}, and later \citet{goldreich1969some}, used a metaphor
of beetles crawling around on the surface of the globe, thereby shifting the moment
of inertia and causing TPW. Within this metaphor, our analysis addresses the
number, speed, and size of the beetles. As the Rayleigh number increases,
we predict more beetles, which move more quickly, but are smaller.

In Chapter~\ref{chap:free_surface} we analyzed numerical methods for simulating
viscous flows with a free surface boundary condition. Free surface boundary conditions
are needed for simulating many tectonic and geomorphologic settings, and
allow for the computation of gravity and moment of inertia perturbations 
in models with arbitrary viscosity structures. 
We explained the so-called ``drunken sailor'' numerical instability that
has afflicted numerical models with free surfaces in terms of the spectrum of relaxation
times for the system. This relaxation time spectrum can be found by solving a generalized eigenvalue problem.
Using this framework, we showed that the commonly used quasi-implicit stabilization
scheme works by lengthening the relaxation times, allowing for longer timesteps.

Our spectral analysis also allows for the construction of a new timestepping
scheme that is rooted in nonstandard finite differences. It is first order accurate,
and allows for much larger timestep sizes than the forward Euler scheme 
(if the numerical analyst is willing to forgo accuracy in the shortest timescales of the system).

In Chapter~\ref{chap:bayesian_plate_reconstruction} we proposed a new method
for analyzing paleomagnetic apparent polar wander paths. Commonly used methods
such as spline fits and running means provide smooth, age progressive APW paths
from paleomagnetic poles, but it is difficult to know what smoothing parameters
are appropriate, or how to incorporate uncertainties in age and position of the poles.
We proposed an extention of the paleomagnetic Euler pole method which uses Bayesian
Markov chain Monte Carlo methods to address these difficulties.
This approach naturally allows for the incorporation of uncertainties in the input data,
and automatically provides uncertainties in the model parameters for which we invert.
Furthermore, the forward model is rooted in the kinematic model used
for the description of plate motions, allowing us to estimate past plate speeds
and their uncertainties.

We applied our Bayesian PEP method to paleomagnetic poles from the 
Mesoproterozoic Keweenawan Midcontinent Rift zone, which has been interpreted to
imply extremely fast plate speeds. Our inversions find that the implied plate
speeds for Laurentia were between 22.9-39.1 cm/yr at a 95\% credible interval.
These speeds are significantly faster than the highest plate speeds in
models of Cenozoic plate motions. True polar wander, which has the potential
to be much faster than plate speeds, is one possible explanation for such high
rates.

Future models for the inversion of past plate motions could include TPW as an
explicit model parameter. Since TPW and plate motions are both described by
finite rotations around Euler poles, it is very difficult to distinguish them,
and there should be significant tradeoffs between the APW path being explained
mostly by plate motions and mostly by TPW. 
There is hope for distinguishing them, however,
by regularization via the prior distributions,
and by the inclusion of more continental blocks in the inversion.
