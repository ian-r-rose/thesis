% This sample file is dedicated to the public domain.
\documentclass[12pt]{myucthesis}

%\nofiles
% The above command prevents latex from writing its auxiliary
% files. This is useful if you want to manually tweak them before you
% generate your final PDF.


% Page layout. The fancyhdr package may complain about the need for a
% larger headheight, depending on how long chapter titles are; if left
% unspecified in the geometry setup, it defaults to 12pt. The
% "showframe" option causes the geometry package (version >= 5.0) to
% show a frame around the margins on every page, which is great for
% checking that you don't overflow anywhere.

%\usepackage[letterpaper,includehead,margin=1in,headheight=15pt,showframe]{geometry}
\usepackage[letterpaper,includehead,margin=1in,headheight=15pt]{geometry}
\usepackage{fancyhdr}
\pagestyle{fancyplain}
\lhead[\fancyplain{\thepage}{\thepage}]{\fancyplain{}{\scshape\rightmark}}
\rhead[\fancyplain{}{\scshape\leftmark}]{\fancyplain{\thepage}{\thepage}}
\chead{}
\cfoot{}
\lfoot{}
\rfoot{}


% Bibliography stuff:

\newcommand{\newblock}{\par} % need this for some natbib internal bug
\usepackage{natbib}
%\citestyle{aa}
\bibliographystyle{elsarticle-harv}
%\setlength{\bibsep}{0ex} % single-space entries
\def\bibpreamble{\addcontentsline{toc}{chapter}{Bibliography}} % get a good TOC entry


% Other setup:

\usepackage[T1]{fontenc} % see http://tinyurl.com/67zdxwf
%\usepackage{ae}
%\usepackage{aecompl}
\usepackage{lmodern}
\usepackage[colorlinks,urlcolor=blue,citecolor=blue,linkcolor=blue,pdfusetitle]{hyperref}
\usepackage{pdflscape} % allows landscape-oriented figures with PDF page rotation
\usepackage{mymacros,amsmath,amssymb,graphicx}
\usepackage{mydeluxetable} % deluxetable customized to play well with ucthesis
\usepackage{longtable} % allow long tables
\usepackage{booktabs} % needed by pandas latex output
\usepackage{appendix} % needed for appendices in each chapter, if required.


\begin{document}
\ssp % single spacing
\hypersetup{pageanchor=false}
\title{True polar wander on convecting planets}
\author{Ian Robert Rose} % must match BearFacts!
\degreesemester{Summer}
\degreeyear{2016}
\degree{Doctor of Philosophy}
\numberofmembers{3}
\chair{Professor Bruce Buffett}
\othermembers{
Professor Nicholas Swanson-Hysell \\
Professor Philip Marcus
}
\field{Earth and Planetary Science}
\campus{Berkeley}

\maketitle
\copyrightpage

\begin{abstract}
Rotating planets are most stable when spinning around their maximum moment of inertia, 
and will tend to reorient themselves to do so. 
Geological activity redistributes mass in the planet, making the moment of inertia a function of time.
As the moment of inertia of the planet changes, the spin axis shifts with respect to
a mantle reference frame in order to maintain rotational stability.
This process is known as true polar wander (TPW). 
Of the processes that contribute to a planet's the moment of inertia,
convection in the mantle generates the largest and longest-period
fluctuations, with corresponding shifts in the spin axis.
True polar wander has been hypothesized to explain several geographic features on planets and moons in our solar system.
On Earth, TPW events have been invoked in some interpretations of paleomagnetic data.
Large swings in the spin axis could have enormous ramifications for paleogeography,
paleoclimate, and the history of life.

Although the existence of TPW is well-verified, it is not known whether its rate and
magnitude have been large enough for it to be an important process in Earth history.
If true polar wander has been sluggish compared to plate tectonic speeds, 
then it would be difficult to detect and its consequences would be minor.
Herein I investigate rates of true polar wander on convecting planets using scaling, numerics,
and inverse problems.

I perform a scaling analysis of TPW on a convecting planet, identifying a minimal
set of nondimensional parameters which describe the problem. The primary nondimensional numbers
that control the rate of TPW are the ratio of centrifugal to gravitational forces ($m$)
and the Rayleigh number ($\mathrm{Ra}$). The parameter $m$ sets the size of the rotational bulge,
which determines the amount of work that needs to be done to move the spin axis.
The Rayleigh number controls the size, distribution, and rate of change of
moment of inertia anomalies, all of which affect the rate of TPW.
I find that the characteristic size of moment of inertia anomalies decreases with
higher $Ra$, but that the characteristic response time for TPW also decreases.
These two effects approximately cancel. However, the orientation of the principal axes of
the moment of inertia becomes less stable to perturbations at high $\mathrm{Ra}$, thereby increasing the rate of TPW.
Overall, I find that a more vigorously convecting planet (one with a higher $\mathrm{Ra}$)
is more likely to experience large TPW events.
If early Earth had more vigorous convection, it may have experienced
more TPW than present-day Earth.

Flow induced by density anomalies in the mantle deflects free surfaces at the surface
and the CMB, and the mass anomalies due to these deflections contribute to the moment of inertia.
A full accounting of the moment of inertia anomalies must include these surface effects.
Numerical models of mantle convection with a free surface have suffered from numerical
sloshing instabilities. I analyze the sloshing instability by constructing a generalized
eigenvalue problem for the relaxation time spectrum. The minimum relaxation time of the spectrum sets the
maximum stable timestep. This analysis gives the first quantitative explanation for
why existing techniques for stabilizing geodynamic simulations with a free surface work. 
I also use this perspective to construct an alternative stabilization scheme based
on nonstandard finite differences. This scheme has a single parameter, 
given by an estimate of the minimum relaxation time, and allows for still larger timesteps.

Finally, I develop a new method for analyzing apparent the polar wander (APW) paths
described by tracks of paleomagnetic poles. Existing techniques, such as spline fits
and running means, do not fully account for the uncertainties in the position
and timing of paleomagnetic pole paths. Furthermore, they impose regularization
on the solution, and the resulting uncertainties are difficult to interpret.
Our technique is an extension of paleomagnetic Euler pole (PEP) analyisis.
I invert for finite Euler pole rotations that can reproduce APW paths
within a Bayesian Markov chain Monte Carlo (MCMC) framework. This allows us
to naturally include uncertainties in age and position, and provides error estimates
on the resulting model parameters. Regularization can be accomplished via
physically motivated choices for the parameters' prior probability distribtions.

I applied the Bayesian PEP technique to the Mesoproterozoic Laurentian APW track,
which primarily comes from the Keweenawan Midcontinent Rift zone. I fit
the track with one and two Euler rotations. Both inversions did a good job
of reproducing the Keweenawan track, though the two Euler pole inversion
has a closer fit. I find that the implied Laurentian plate speeds exceeds
22.9 cm/yr at the 95\% confidence level. These speeds are significantly faster than
Cenozoic plate speeds, and could be explained by either faster plate speeds in
the Proterozoic or a TPW event.


\end{abstract}

\hypersetup{pageanchor=true}
\begin{frontmatter}

\begin{dedication}
\null\vfil
{\large
\begin{center}
For my family
\end{center}}
\null\vfil
\end{dedication}

\tableofcontents
\listoffigures % optional
\listoftables % optional

% If using code.sty, can also add:
%% \listofcodes
%% \addcontentsline{toc}{chapter}{List of Code Examples}

\begin{acknowledgements}

% Feel free to modify or remove this acknowledgment:
This dissertation was typeset using the
\href{https://github.com/pkgw/ucastrothesis}{\textsf{ucastrothesis}}
\LaTeX\ template.

\end{acknowledgements}
\end{frontmatter}

% This sample file is dedicated to the public domain.
\chapter{Introduction}
\label{c.intro}

Abstract.

\section{Introduction}


\section{Feelings on My Thesis}

\input{intro/processed.tex}

\section*{Acknowledgments}

\include{tpw_rate/tpw_rate_chapter}
\include{free_surface/free_surface_chapter}
\include{bayesian_plate_reconstruction/bayesian_plate_reconstruction_chapter}
\chapter{Conclusion and outlook}
\label{chap:conclusion}

The work in this dissertation has approached the topic of true polar wander
from several directions, including scaling, numerics, and data analysis.

In Chapter~\ref{chap:tpw_rate} we analyzed TPW from the perspective of fluid dynamics.
The rotating planet is most stable when the spin axis coincides with the axis of the largest moment of inertia.
TPW is primarily a balance between the generation of anomalies in the moment of inertia tensor
via convection and their decay via movement of the spin axis.
Our scaling showed that the primary nondimensional numbers controlling TPW are the Rayleigh number $\mathrm{Ra}$,
and the Froude number $m$, defined by the ratio of centrifugal to gravitational forces.

The Froude number sets the size of the rotational bulge, which acts
as the brakes on TPW. The dependence on the Rayleigh number is more complicated:
at high $\mathrm{Ra}$ the flow becomes more chaotic, and the total power in the degree-two
part of the density field (which is the part that controls TPW) goes down.
However, at high $\mathrm{Ra}$ the characteristic response time for TPW also goes down,
and these two effects largely cancel. Additionally, at higher $\mathrm{Ra}$,
the perturbations to the moment of inertia can produce larger angular differences
between the axis of the maximum moment of inertia and the spin axis,
which results in faster rates of TPW. The net effect of higher $\mathrm{Ra}$,
and the correspondingly more vigorous convection, is that TPW events become more likely.
\citet{gold1955instability}, and later \citet{goldreich1969some}, used a metaphor
of beetles crawling around on the surface of the globe, thereby shifting the moment
of inertia and causing TPW. Within this metaphor, our analysis addresses the
number, speed, and size of the beetles. As the Rayleigh number increases,
we predict more beetles, which move more quickly, but are smaller.

In Chapter~\ref{chap:free_surface} we analyzed numerical methods for simulating
viscous flows with a free surface boundary condition. Free surface boundary conditions
are needed for simulating many tectonic and geomorphologic settings, and
allow for the computation of gravity and moment of inertia perturbations 
in models with arbitrary viscosity structures. 
We explained the so-called ``drunken sailor'' numerical instability that
has afflicted numerical models with free surfaces in terms of the spectrum of relaxation
times for the system. This relaxation time spectrum can be found by solving a generalized eigenvalue problem.
Using this framework, we showed that the commonly used quasi-implicit stabilization
scheme works by lengthening the relaxation times, allowing for longer timesteps.

Our spectral analysis also allowed for the construction of a new timestepping
scheme that is rooted in nonstandard finite differences. It is first order accurate,
and allows for much larger timestep sizes than the forward Euler scheme 
(if the numerical analyst is willing to forgo accuracy in the shortest timescales of the system).

In Chapter~\ref{chap:bayesian_plate_reconstruction} we proposed a new method
for analyzing paleomagnetic apparent polar wander paths. Commonly used methods
such as spline fits and running means provide smooth, age progressive APW paths
from paleomagnetic poles, but it is difficult to know what smoothing parameters
are appropriate, or how to incorporate uncertainties in age and position of the poles.
We proposed an extension of the paleomagnetic Euler pole method which uses Bayesian
Markov chain Monte Carlo methods to address these difficulties.
This approach naturally allows for the incorporation of uncertainties in the input data,
and automatically provides uncertainties in the model parameters for which we invert.
Furthermore, the forward model is rooted in the kinematic model used
for the description of plate motions, allowing us to estimate past plate speeds
and their uncertainties.

We applied our Bayesian PEP method to paleomagnetic poles from the 
Mesoproterozoic Keweenawan Midcontinent Rift zone, which has been interpreted to
imply extremely fast plate speeds. Our inversions find that the implied plate
speeds for Laurentia exceeded 22.9 cm/yr at a 95\% credible interval.
This speed is significantly faster than the highest plate speeds in
models of Cenozoic plate motions. True polar wander, which has the potential
to be much faster than plate speeds, is one possible explanation for such high rates.


\bibliography{thesis}
\end{document}
